\begin{abstract}
% Abstract length should not exceed 250 words (Current word count: 241)
Gene duplications have been shown to promote evolvability in biology and in computational systems. We use digital evolution to explore \textit{why}; that is, what characteristics of gene duplications increase evolutionary potential?
%Is it their capacity to duplicate meaningful genetic information in the genome, create new genetic material for evolutionary processes to act on, or increase the effective mutation rate? Or, is it some combination of these aspects that enable gene duplications to promote evolvability in a system?
Are duplications valuable because they inflate the effective mutation rate, generating increased amounts of genetic variation?
Or is it that those mutations are clustered together?
Or, is it that the mutations insert genetic material, providing evolution an easy technique to select for longer genomes?
Does the value pertain to the information being duplicated in the genome? If so, is the full structure of duplicated code critical, or would the duplication of functional building blocks be valuable even if rearranged?
Using the Avida Digital Evolution Platform, we experimentally tease apart these aspects in two qualitatively different environments: one where complex computational traits are directly selected, and another where those traits need to be regulated based on current environmental conditions.
We confirm that gene duplications promote evolvability in both static and changing environments.
Furthermore, we find that the primary value of gene duplications comes from their capacity to duplicate existing genetic information within a genome.
Specifically, while duplications that randomize the order of genetic material are valuable, the most useful form of duplication also preserve the structure (and thus information content) of duplicated sequences.
\end{abstract}
