\begin{abstract}
Gene duplications are a well‐recognized engine of evolutionary innovation, yet questions remain about the mechanisms by which they enhance adaptive potential.
Here, we use the Avida digital evolution platform to systematically explore different facets of gene duplication and how they influence adaptive evolution.
By comparing treatments applying a variety of duplicative mutation operators (such as exact duplications, shuffled duplications, or injections of random genetic material), we disentangle the benefits to adaptive evolution from increased mutational supply, clustering of mutational locality, and the propagation of existing genetic material.
Our results reveal that while simply expanding genome size can promote the emergence of simple adaptive traits, it is the reuse of existing genetic information through exact duplication that uniquely facilitates the evolution of complex phenotypes.
Tracing the ancestry of individual genetic sites, we find coding sites for novel phenotypic traits to be concentrated in previously-duplicated genomic regions, suggesting that duplications may act to potentiate discovery of complex adaptive traits.
Together, these findings underscore the necessity of integrative approaches in studying evolution that incorporate understanding of how mutational processes encode and alter genetic information.
% @CAO: This last line feels outside the scope of this paper; I think it's a fine thing to say in the body, but the abstract should end with a line about what our results mean in a broader context.  Perhaps most specific follow-up topics that can now be addressed, or maybe a link to the EC side of things where we can apply this ideas for more rapid problem solving.
% @MAM: agreed
\end{abstract}
