\section*{Abstract}

In numerous instances across the record of natural history, gene duplication has been tied both to adaptive evolution of novel phenotypic functionality and to accumulation of neutral subdivisions in existing functionality.
On this basis, gene duplication is broadly argued as a key factor in the evolution of organismal complexity.
However, practical challenges exist in applying direct experimental approaches to assess impact on aggregate organismal complexity by such neo- and subfunctionalization processes, respectively, over extended evolutionary time.
To test this question, we turn to the Avida platform for digital evolution, which enables comparison of arbitrary mutational processes over long durations in populations of virtual organisms.
Across evolutionary trials, we find that duplicative inserts increase both evolved functional and structural complexity, but that the mechanistic significance of genetic information within duplications differs between these effects.
Adaptive phenotypes arose more readily when duplications carried genetic information, compared to treatments with genetic content of duplications ablated.
While increased genome size alone promotes emergence of simple adaptive traits, however, we find that gene duplication uniquely facilitates evolution of \textit{complex} adaptive phenotypes.
Lineage histories further indicate a direct effect, with duplicated regions exhibiting elevated likelihood to code for complex phenotypic traits.
With respect to structural complexity, we found genetic content of duplications to significantly increase genome size.
Unlike effects on functional complexity of phenotypic traits, however, increased genome size alone produces similar quantities of genetic coding sites as full-fledged duplication processes.
The integrative perspective on long-term evolutionary trajectories from our experiments provides positive evidence for aggregate consequences of gene duplication on promoting complexity under both neutral and adaptive framings of evolution of complexity.
However, the adaptive effect on functional complexity is particularly exceptional beyond what can be achieved through other mechanisms promoting genome length, while neutral effects are commensurate with net increases in genome length.
The integrative perspective on long-term evolutionary trajectories afforded by reported experiments speaks directly to open questions as to the bottom-line role of gene duplication under both adaptive and neutral framings of biological complexity.
In sum, our findings show that information content of duplicative inserts can both promote genome expansion and raise the upper ceiling on functional complexity of adaptive traits, but suggest a more ancillary role in driving structural complexity of genome coding sites.
