\begin{abstract}
Gene duplication events are widely recognized as a key factor in facilitating the evolution of organismal complexity.
Mirroring broader debate around adaptation- versus contingency-driven explanations for the evolutionary origins of biological complexity, gene duplication outcomes are typically framed in terms of neo- and sub-functionalization scenarios.
In the former, duplicated genetic material catalyzes novel functionality; in the latter, it is co-opted to elaborate existing functionality.
Although examples of both scenarios are widespread in natural history, practical constraints have limited direct experimental investigation of the relationship between gene duplication and organismal complexity.
Using the Avida platform for digital evolution, we show that while merely expanding genome size can promote the emergence of simple adaptive traits, gene duplication uniquely facilitates the \textit{de novo} evolution of complex adaptive phenotypes.
Tracing the ancestry of individual genetic sites, we find that coding sites for novel phenotypic traits are concentrated in previously duplicated genomic regions.
We then harness the unique capabilities of \textit{in silico} model systems to compare evolutionary outcomes under degraded variants of full slip-duplication.
This ablative analysis confirms that the observed adaptive potentiation indeed arises from the duplication of existing genetic information.
In contrast to purely neutral framings of biological complexity, our results support gene duplication events as a contributing factor in adaptive origins of complex traits.
\end{abstract}
